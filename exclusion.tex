\section{Selection Factors --- Exclusion}

% so that it starts with f, specify counter value
\setcounter{subsection}{5}

\subsection{Overly Specific}

This emoji sequence is suitable to indicate all forms of neurodiversity. It is not
overly specific to any particular form of neurodiversity.

\subsection{Open-ended}

Neurodiversity is also often represented by a rainbow-colored mathematical infinity
symbol, see figure~\ref{fig:infinity} for an example.

The existing \textbf{INFINITY} \texttt{U+267E U+FEOF} specifies the symbol is enclosed
within a circle or square, making it difficult to use with a rainbow modifier.

Using a rainbow infinity would probably require a new codepoint assignment to properly avoid that problem.

The stroke thickness may also cause issues with multiple colors at small sizes.

\subsection{Already Representable}

Some people are already using \textbf{BRAIN} \texttt{U+1F9E0} to represent neurodiversity, see
figure~\ref{fig:rebel}. It works for the present but only indicates neurodiversity for those
who already understand the context.

Some people are already using \textbf{JIGSAW} \texttt{U+1F9E9} to represent autism, see
figure~\ref{fig:covfefe}. When used in the context of autism it has trademark issues, and it
excludes neurodiversity that is not autistic.

\subsection{Logos, Brands, UI Icons, Specific People, Deities}

To the best of my knowledge, the Rainbow Brain is not exclusive to any specific brand or
person.

The blue puzzle piece that Twitter uses for \textbf{JIGSAW} \texttt{U+1F9E9} however is very
similar to a trademark owned by Autism Speaks. See figure~\ref{fig:aspeaks}. Encouraging the
use of a rainbow colored brain for neurodiversity would help avoid trademark issues.

\subsection{Transient}

As awareness of autism and other neurodivergent traits increase, the neurodiversity community
will continue to grow, especially online where our social awkwardness is less of an issue. I
expect use of this emoji to be of value for as long as people still populate the planet.

\subsection{Faulty Comparison}

In part, this request is in response to the additiom of the \textbf{JIGSAW} \texttt{U+1F9E9}
emoji but only in the respect that an emoji for neurodiversity is needed, and the use of a
puzzle piece to fill that need violates an existing trademark as well as identifying with a
specific group that many in the neurodiverse community do not wish to be associated with.

\subsection{Exact Images}

This emoji sequence does not require an exact image. The existing \textbf{BRAIN} \texttt{U+1F9E0}
glyph used by existing emoji fonts modified to use a rainbow is sufficient.
