\section{Selection Factors --- Exclusion}

% so that it starts with f, specify counter value
\setcounter{subsection}{5}

\subsection{Overly Specific}

The emojis proposed here are suitable to indicate all forms of neurodiversity. They are not
overly specific to any particular form of neurodiversity.

\subsection{Open-ended}

The two proposed emojis here are the most frequently used pictographs I have seen to indicate
neurodiversity within the neurodiversity community.

Outside the neurodiversity community, and by some members of the autistic community, a jigsaw
puzzle piece is sometimes used. However the puzzle piece is closely tied to a specific organization
and is in fact a trademark of that organization, making it unsuitable for general use.

\subsection{Already Representable}

Some people are already using \brainemoji{} to represent neurodiversity.
It works for the present but only indicates neurodiversity for those
who already understand the context. Far more often, that emoji is used to indicate intelligence
or thinking.

Some people are already using \jigsawemoji{} to represent autism.
When used in the context of autism it has trademark issues, and it
excludes neurodiversity that is not autistic.

\subsection{Logos, Brands, UI Icons, Specific People, Deities}

To the best of my knowledge, the Rainbow Brain and Rainbow Infinity Loop are not exclusive to
any specific brand or person.

I did find the infinity loop in use with some Tarot card decks, but without a rainbow spectrum.

The blue puzzle piece that Twitter uses for \jigsawemoji{} however is very
similar to a trademark owned by Autism Speaks. See figure~\ref{fig:aspeaks}. Encouraging the
use of a rainbow colored brain or rainbow infinity loop for neurodiversity would help avoid
trademark issues.

\subsection{Transient}

As awareness of autism and other neurodivergent traits increase, the neurodiversity community
will continue to grow, especially online where our social awkwardness is less of an issue. I
expect use of these emojis to be of value for as long as people still populate the planet.

\subsection{Faulty Comparison}

I do not believe this is applicable.

\subsection{Exact Images}

An exact image is not needed for either of these proposed emojis. Emoji font creators have plenty
of artistic license when creating the glyphs to represent them.
