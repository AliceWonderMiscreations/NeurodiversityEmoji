\section{Selection Factors --- Special Consideration}

The emoji \jigsawemoji{} was recently added to the Unicode Emoji list.

The organization \myhref{https://www.autismspeaks.org/}{Autism Speaks} uses a blue puzzle
piece for their logo, see figure~\ref{fig:aspeaks}. Twitter uses a very similar blue
puzzle piece for the \jigsawemoji{} emoji, see figure~\ref{fig:twitterJigsaw}

The two are very similar and there is some considerable fear that people outside of the
autism community will start to use that emoji when referencing autism.

Within the autism community (people who are actually autistic themselves) there is a lot of
resentment towards that particular group as well as resentment towards using a puzzle piece
to represent us.

For information on why many (though admittedly not all) within the autistic community dislike
the Autism Speaks organization, see
\myhref{https://neurodivergentrebel.com/2018/03/14/why-autistic-people-generally-dislike-autism-speaks/}{Why Autistic People Generally Dislike Autism Speaks}.

For information on why many within the autistic community do not want to be represented by a
puzzle piece, see
\myhref{https://learnfromautistics.com/the-problem-with-the-autism-puzzle-piece/}{The Problem with the Autism Puzzle Piece}.

Whether or not that organization deserves that resentment from us though is not the issue.
Regardless of those things, it is not a good idea for an emoji that is very close to the
trademark of a specific autism related organization to be used for autism in the general
sense.

The proposed neurodiversity emojis would give an alternative that will not be confused
with the trademark of a specific autism related organization while at the same time giving our
community an emoji to use that we specifically chose to identify with.

The proposed neurodiversity emojis avoids the potential misuse of the \jigsawemoji{}
emoji in the autism context it was not intended for.

