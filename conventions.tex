\section*{Conventions}

This proposal is attempting to follow the proposal guidelines specified at
\url{https://unicode.org/emoji/proposals.html} using the same section numbering
scheme used in the form specification there.

Section Six, \textit{`Other Information'} in the proposal form template, has been renamed to
\textit{`Selection Factors --- Special Consideration'}.

Sometimes I over-explain thing. It is an autism trait often known as `data dumping' and
is often frowned upon by those who are neurotypical. I have tried very hard not to do
it here, but I ask for an ounce of grace if and when I over-explain something.

\subsection*{Existing Emojis}

Existing emojis used in this document will be referenced using upper case bold text for the CLDR short name
followed by an image of the emoji.

Unfortunately \TeX{}Live does not directly support color emoji fonts. I was forced to
create the following hack:

\begin{verbatim}
\newcommand{\emojipedia}[1]{$\vcenter{\hbox{\includegraphics[height=\baselineskip]{emojipedia/#1}}}$}
\end{verbatim}

It works visually but it means the emoji can not be copied into a clipboard as the
Unicode codepoint or codepoint sequence. In other words, it is dirty. Hopefully a
mechanism for using color emoji fonts with \TeX{}Live and \texttt{pdflatex} that
allows copying the emoji as text will be developed.

PNG images for the emojis were obtained from \url{https://emojipedia.org/} and actually
represent a variety of color emoji fonts, I picked what I liked.

\subsection*{Hyperlinks}

Hyperlinks where the text of the hyperlink is not the URL will have the URL specified
in a footnote.

\subsection*{Figures}

With the exception of the concept are for the proposed emojis at the top of
this document, all figures appear in Appendix~\ref{apx:figures}. This was done because
some of the figures are referenced more than once, and to make it easier to note that
the images in the figures are used under the Fair Use doctrine so that it is very
clear the copyright to those images does not belong to me. I like to be very clear
when it comes to intellectual property.

\subsection*{Community Participation}

I, Alice Wonder, am only a \emph{member} of the Autistic / Neurodiverse community.

This document was created on a
\myhref{https://github.com/AliceWonderMiscreations/NeurodiversityEmoji}{Public Github}
where any member of the community is free to either create issues or submit pull
requests.

Community participation has been sought at both Twitter and Facebook \emph{er, will be...}
before submission of this proposal to the Unicode Consortium to make sure this proposal is
congruent with the those the emoji is intended to represent.

\subsection*{Document Compile Instructions}
This subsection to be removed before final submission.

For those familiar with \LaTeX{} who wish to submit a pull request, this PDF document
should compile with any somewhat recent version of
\myhref{https://tug.org/texlive/}{\TeX{}Live} as long as you have the
\myhref{https://ctan.org/tex-archive/fonts/LuxiMono/}{LuxiMono} font pack installed. I believe
all the other needed \LaTeX{} packages are part of a standard \TeX{}Live install.

The \LaTeX{} files in the github repository are intended to be compiled into a PDF document
using the following method:

\begin{verbatim}
pdflatex proposal.tex && pdflatex proposal.tex && pdflatex proposal.tex
\end{verbatim}

Do not include updated PDF if your pull request, but please do check that and TeX changes
do not break compiling of the document.

For those not familiar with \LaTeX{} just open an issue with the github project with your
suggested changes.

