\section*{Conventions}

This proposal is attempting to follow the proposal guidelines specified at
\url{https://unicode.org/emoji/proposals.html} using the same section numbering
scheme used in the form specification there.

Section Six, \textit{`Other Information'} in the proposal form template, has been renamed to
\textit{`Selection Factors --- Special Consideration'}.

Sometimes I over-explain thing. It is an autism trait often known as `data dumping' and
is often frowned upon by those who are neurotypical. I have tried very hard not to do
it here, but I ask for an ounce of grace if and when I over-explain something.

\subsection*{Existing Emojis}

Existing emojis used in this document will be referenced using upper case bold text for the CLDR short name
followed by an image of the emoji.

Unfortunately \TeX{}Live does not directly support color emoji fonts. I was forced to
use a hack that inserts the emojis as images.

I apologize if that screws up automated software that generates emoji use statistics.
\LaTeX{} is the only word processor I ever was able to figure out how to get it to do
what I want. Well, um,
until now when I wanted to insert emojis... ;)

\subsection*{Hyperlinks}

Hyperlinks where the text of the hyperlink is not the URL will have the URL specified
in a footnote.

\subsection*{Figures}

With the exception of the concept art for the proposed emojis at the top of
this document, all figures appear in Appendix~\ref{apx:figures}.
This makes it easier to note that the images in the figures are used under the Fair
Use doctrine so that it is very clear the copyright to those images does not belong
to me. I like to be very clear
when it comes to intellectual property.

\subsection*{Community Participation}

I, Alice Wonder, am only a \emph{member} of the Autistic / Neurodiverse community.

This document was created on a
\myhref{https://github.com/AliceWonderMiscreations/NeurodiversityEmoji}{Public Github}
where any member of the community is free to either create issues or submit pull
requests.

Community participation has been sought at both Twitter and Facebook
before submission of this proposal to the Unicode Consortium to make sure this proposal is
congruent with the those the emoji is intended to represent.

Feedback from members of the community on both Twitter and Facebook has been positive. It
is only a very small sampling of the larger community, but I did my best to contact as
many members of the community as I could.

