\section*{Conventions} %\addcontentsline{toc}{section}{Conventions}

This proposal is attempting to follow the proposal guidelines specified at \url{https://unicode.org/emoji/proposals.html} using the same section numbering scheme used in the form specification there.

Uppercase bold text indicates the CLDR short name of an emoji. Monospace (typewriter) text beginning with \texttt{U+} indicates a Unicode codepoint. When referencing a CLDR short name for an existing emoji, the first time it is referenced it will always be accompanied with the Unicode codepoint.

Hyperlinks where the text of the hyperlink is not the URL will have the URL specified in a footnote.

Section Six, `Other Information' in the proposal form template, has been renamed to `Selection Factors --- Special Consideration'.

With the exception of an example of the proposed emoji which is required at the top of the document, all figures appear in Appendix~\ref{apx:figures}.

Sometimes I over-explain thing. It is an autism trait often known as `data dumping' and is often frowned upon by those who are neurotypical. I have tried very hard not to do it here, but I ask for an ounce of grace if and when I over-explain something.

\subsection*{Community Participation}

I, Alice Wonder, am only a \emph{member} of the Autistic community.

This document was created on a \myhref{https://github.com/AliceWonderMiscreations/NeurodiversityEmoji}{Public Github} where any member of the Autstic community is free to either create issues or submit pull requests.

Community participation has been sought at both Twitter and Facebook \emph{er, will be...} before submission of this proposal to the Unicode Consortium to make sure this proposal is congruent with the those the emoji is intended to represent.

Document compile instructions to be removed before final submission.

For those familiar with \LaTeX{} who wish to submit a pull request, this PDF document should compile with any somewhat recent version of \myhref{https://tug.org/texlive/}{\TeX{}Live} as long as you have the \myhref{https://ctan.org/tex-archive/fonts/LuxiMono/}{LuxiMono} font pack installed. I believe all the other needed \LaTeX{} packages are part of a standard \TeX{}Live install.

The \LaTeX{} files in the github repository are intended to be compiled into a PDF document using the following method:

\begin{verbatim}
pdflatex proposal.tex && pdflatex proposal.tex && pdflatex proposal.tex
\end{verbatim}

For those not familiar with \LaTeX{} just open an issue with the github project with your suggested changes.

