\section{Selection Factors --- Inclusion}

\subsection{Compatibility}
Some members of the autistic / neurodiversity community currently use \textbf{BRAIN} \texttt{U+1F9E0}
to indicate neurodiversity, as it is the closest emoji we currently have that expresses our
emerging demographic identity. See figure~\ref{fig:rebel}.

\subsection{Expected Usage Level}

\subsubsection{Frequency}

To be written

\subsubsection{Multiple Usages}

Not all people who are neurodiverse opposed to neurotypical are autistic. That is why
\textbf{NEURODIVERSITY} is used for the CLDR short name.

This is meant for them as well if they wish to use it, anyone who is part of the larger
neurodiverse community.

We expect this emoji will largely be used by the autistic community but have experienced
much exclusion ourselves and do not wish to exclude members of the neurodiverse community
who are not autistic.

Furthermore, some who are autistic prefer to identify as neurodiverse simply because of
the social stigmas that often accompany the word autistic, such as `autistic behavior' and
the official labeling of autism as a mental health disorder rather than as valuable diversity
in how to perceive things.

Brain scans and instructive artwork for brains often use colors to indicate different things,
whether it is neurotypical or not, see figure \ref{fig:brainparts}. It would not be surprising
if some people in the field of neuroscience used this emoji sequence outside of the context
of neurodiversity.

\subsubsection{Use in Sequences}

This emoji is a proposed sequence. It may just be lack of imagination, but I am not able to
conceive of any scenarios where additional sequences would be needed.

\subsubsection{Breaking New Ground}

There currently are not any emojis that exist for the purpose of identifying diversity inside
a person. There are emojis for ethnic differences, gender differences, career differences,
even hair style differences.

Nothing seems to exist for differences in people that are fundamentally internal yet are a
huge part of who they are.

An argument could be made for the \textbf{NERD FACE} \texttt{U+1F913} and the
\textbf{RAINBOW FLAG} \texttt{U+1F3F3 U+200D U+1F308} having broken that ground already.

\subsection{Image Distinctiveness}

A rainbow is frequently used as a visual representation of a spectrum. A rainbow itself is
created by diffraction of white light, allowing our eyes to interpret the individual
wavelengths that make up the visible spectrum of light and see the beautiful diversity that
actually exists within white light that we were unable to perceive without the diffraction of
light.

It should not be too difficult for people to figure out the rainbow is acting as a diversity
adjective to the \textbf{BRAIN} emoji and understand the neurodiversity context.

A rainbow flag is often used as a symbol of pride by the LGBTQIA and sometimes rainbow colors
and patterns are used outside of the context of a flag to indicate LGBTQIA pride. However the
\textbf{RAINBOW FLAG} sequence \texttt{U+1F3F3 U+200D U+1F308} is more frequently used for
LGBTQIA pride. I do not believe there will be much confusion.

\subsection{Frequency Request}

A search on Twitter for the \textbf{BRAIN} emoji shows frequent, though not exclusive, use of
that particular emoji in the context of neurodiversity.

A search on Twitter for the \textbf{JIGSAW} emoji showed some usage of the emoji in the context
of neurodiversity, see figure~\ref{fig:covfefe}.

With two different emojis being used in the context of neurodiversity that are not intended to
convey neurodiversity, we believe there is sufficient demand for an emoji designed to represent
our demographic.
