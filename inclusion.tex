\section{Selection Factors --- Inclusion}

\subsection{Compatibility}
Not applicable.

\subsection{Expected Usage Level}

\subsubsection{Frequency}

A search on Twitter revealed several users using \brainemoji{} emoji or a brain in
their avatar in the context of neurodiversity:
\begin{itemize}
  \item \myhref{https://twitter.com/NeuroRebel}{Neurodivergent Rebel}
  \item \myhref{https://twitter.com/diffbrains}{Different Brains}
\end{itemize}

A search on Twitter for neurodivergent / autism related hashtags revealed several users with a rainbow
infinity loop in their avatar:

\begin{itemize}
  \item \myhref{https://twitter.com/AutisticPriest}{Autistic Priest}
  \item \myhref{https://twitter.com/ChiDeltaWithNOR}{Chris Connor}
  \item \myhref{https://twitter.com/Frances_Larina}{Frances\_Larina}
  \item \myhref{https://twitter.com/meowcarriemeow}{Meow Carrie Meow}
\end{itemize}

A search on Twitter revealed several users using \jigsawemoji{} in the context of autism.

\begin{itemize}
  \item \myhref{https://twitter.com/adspong2015lego}{Adam Spong}
  \item \myhref{https://twitter.com/wjxt4/status/1015279222657617922}{News4JAX Tweet}
  \item \myhref{https://twitter.com/EmiForLove/status/1008835449798975490}{Autistic Pride Day Tweet}
\end{itemize}

It is apparent that users of the Twitter platform do want iconography related to neurodiversity
and autism.


\subsubsection{Multiple Usages}

There is at least one possible use for the color brain emoji beyond neurodiversity.

Brain scans and instructive artwork for brains often use colors to indicate different things,
whether it is neurotypical or not, see figure \ref{fig:brainparts}. It would not be surprising
if some people in the field of neuroscience used this emoji sequence outside of the context
of neurodiversity.


\subsubsection{Use in Sequences}

I believe this is not applicable.

It may be possible to add these emojis by using a sequence to describe them, such as how
\rainbow{} is used to as a sequence modifier to create the
\prideflag{} emoji.

The technical details of whether to use sequences to add support for these emojis or define
codepoints is beyond the scope of this proposal, that is a Unicode Consortium decision.

\subsubsection{Breaking New Ground}

These emojis \emph{may} break some new ground.

There currently are not any emojis that exist for the purpose of identifying diversity inside
a person. There are emojis for ethnic differences, gender differences, career differences,
even hair style differences.

Nothing seems to exist for differences in people that are fundamentally internal yet are a
huge part of who they are.

An argument could be made for the \nerdface{} and the
\prideflag{} emojis having broken that ground already.

\subsection{Image Distinctiveness}

A rainbow is frequently used as a visual representation of a spectrum. A rainbow itself is
created by diffraction of white light, allowing our eyes to interpret the individual
wavelengths that make up the visible spectrum of light and see the beautiful diversity that
actually exists within white light that we were unable to perceive without the diffraction of
light.

With the proposed Rainbow Brain emoji, it should not be too difficult for people to figure out
the rainbow is acting as a diversity adjective for the brain.

With the proposed Rainbow Infinity Loop emoji, it may require some context initially for people
not already familiar with the Rainbow Infinity Loop to understand the meaning. The current use
of that pictograph by many users already should allow many to know what it means right away.

A rainbow flag is often used as a symbol of pride by the LGBTQIA community and sometimes rainbow colors
and patterns are used outside of the context of a flag to indicate LGBTQIA pride. However the
\prideflag{} is more frequently used for
LGBTQIA pride. I do not believe there will be much confusion though there could be some.

\subsection{Completeness}

In addition to the Rainbow Brain and the Rainbow Infinity Loop, I am aware of two other pictographs
that are sometimes used to indicate neurodiversity in general and/or autism.

A gold heart is sometimes used to indicate Autistic Unity. I \emph{believe} that falls under the
categorization of furthering a cause, which is specified as an invalid reason to request an emoji.

Those who want to use a gold heart also can already do so using \goldheart{}.

Sometimes a puzzle piece is used to indicate autism specifically. The source of this as a pictograph
for autism is due to the organization Autism Speaks using a puzzle piece as their logo, see
figure~\ref{fig:aspeaks}.

Those who wish to use jigsaw puzzle piece to indicate autism can already do so using \jigsawemoji{}
though I would \emph{personally} recommend against it, as it may result in trademark violation when
the emoji font uses a blue puzzle piece for that codepoint.

\subsection{Frequency Request}

Not Applicable, I do not have any data on how often such an emoji is requested either of the
Unicode Consortium or a Unicode member company.
