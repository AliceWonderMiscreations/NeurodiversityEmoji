\section{Images}

The Rainbow Brain concept art in this proposal is based on the Google\textsuperscript{\textregistered}
\myhref{https://www.google.com/get/noto/help/emoji/}{Noto Color Emoji} font. I am not a
graphic artist, I had to start with something.

It does not matter if the emoji font artists use their existing \brainemoji{} glyph
or create a fresh version from scratch, the purpose of the rainbow brain is
sufficiently different from \brainemoji{} that it does not need to inherit the
design of that glyph.

It is probably advantageous for monochrome emoji fonts to NOT base their glyph for
this emoji on their glyph for \brainemoji{} just to provide visual distinction.

The Rainbow Infinity Loop concept in this proposal is directly taken from
\myhref{https://commons.wikimedia.org/wiki/File:Autism_spectrum_infinity_awareness_symbol.svg}{Wikimedia Commons}
with only a canvas change to make the artwork square. Monochrome versions of the glyph
should \emph{probably} just use a monochrome infinity loop.

Monochrome concept art has not been created. I apologize, I am not a graphic artist
so any concept art I create is just that, concept art. Please do not hold that against
this proposal, font creators will do their own thing anyway.

\subsection{Zip File}
A zip file containing the concept art can be downloaded from:
\myhref{https://notrackers.com/NeurodiversityEmojis.zip}{NeurodiversityEmojis.zip}.

The zip archive contains both emoji concepts in the 72x72 size.

The Apache 2.0 license text for the Rainbow Brain emoji is included in the zip file.

\subsection{License}

The Rainbow Brain concept artwork is a derivative of the Noto Emoji \brainemoji{} glyph
which uses \myhref{https://github.com/googlei18n/noto-emoji/blob/master/LICENSE}{Apache License, version 2.0}.
A copy of that license is included in the zip archive.

The Infinity Loop concept artwork is from
\url{https://commons.wikimedia.org/wiki/File:Autism_spectrum_infinity_awareness_symbol.svg}
where it is specified as belonging in the public domain.

\subsection{Document}

The form for Emoji Proposals specifies that images for the proposed emoji be included at
the top of this document in both 18x18 and 72x72 pixels.

The zip archive contains the concept art in 72x72. An 18x18 sample
is not included, that size does not make sense, especially when so many devices have a pixel
density of 2:1 or greater. I did include the concept art scaled to the height of the text
in the document abstract. I hope that is sufficient.

With all due respect, that part of the `Form for Emoji Proposals' needs to be modified. I do
not even really understand what it is after, and several example proposals I looked at did
not adhere to the text of that guideline. See the
\myhref{http://www.unicode.org/L2/L2016/16279-person-meditating.pdf}{\textbf{PERSON MEDITATING}}
sample proposal. The images there do not include 18x18 pixel images.

Is it possible that 18 point and 72 point is what was intended? Those are common font sizes.
However many (most?) emojis are not even remotely recognizable at 18 pixels by 18 pixels. At
least not to my eyes.

