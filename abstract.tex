\begin{abstract}
I am proposing an emoji sequence for \textbf{NEURODIVERSITY}

\begin{figure}[h!]
\caption[NEURODIVERSITY]{\textbf{NEURODIVERSITY}}
\centering
\includegraphics[width=0.333\textwidth]{neurodiversity.png}
\label{fig:neurodiversity}
\end{figure}

Autism (often referred to as Autism Spectrum Disorder) refers to a neurologically
atypical brain often characterized by repetitive behaviors, differences in language
processing, difficulty in social interactions, and differences in how emotions and
empathy are expressed.

The center for disease control estimates that roughly 1 in 68 children is autistic.
Generally boys are more likely to be diagnosed as autistic than girls, though that
may be due to differences in social constructs regarding the expected behavior of
boys and girls resulting is a higher percentage of autistic girls going undiagnosed.

As online social networking makes it easier for autistic individuals to find each
other, our positive identity as a demographic rather than a negative identity as a
disorder is starting to grow.

We believe an emoji to express our neurodiverse identity, an emoji chosen by our
community, is both warranted and will help us continue to identify each other and
grow as a community, enabling us to receive the important peer support we often
lack from society at large.

The primary difference between those who are autistic and those who are not is in
how our brain works. A rainbow repesents the full spectrum of visible light. We
believe a rainbow brain is a logical represention of the neurodiverse spectrum that
makes up our community.

The unicode codepoint \texttt{U+1F9E0} is already used for a brain emoji. The unicide
codepoint \texttt{U+1F308} is already used for a rainbow emoji.

The sequence of codepoints \texttt{U+1F9E0 U+200D U+1F308} could thus be an appropriate
sequence to indicate a neurodiverse brain without the need for assigning a specific
codepoint.
\end{abstract}
